% ----------------------------------------------------------
\chapter{Exemplos}\label{cap:exemplos}
% ----------------------------------------------------------

% Polos em V_{OLF}

% ------------------ MATRIZES DO SISTEMA% ------------------ 

    

    0.572 &  3.217 & 0.005 & 0.189 & 0.017                              & 0.015 &  0.015 & 0.013 & 0.013 & 0.011                            &  0.166 & 0.010 &  4.417 &  0.002 & 0         \\
    1.001 & -5.630 & 0.008 & 0.331 & 0.030                              & 0.026 & 0.026 &  0.022 & 0.022 & 0.019                            & 0.019 &  0.172 & -7.730 & 0.004 &  0      \\
    0 &  0 &  0 &  0 &  0                                               &  0 &  0 &  0 &  0 &  0                                            &  0 &  0 &  0.517 &  0.005 &  0             \\
    0 &  0 &  0 &  0 &  0                                               &  0 &  0 &  0 &  0 &  0                                            &  0 &  0 & -97.622 & 0.260 &  0.001         \\
    0 &  0 &  0 &  0 &  0                                               &  0 &  0 &  0 &  0 &  0                                            &  0 &  0 & -3768.616 & -127.611 & 0.419







    0.572 &  3.217 & 0.005 & 0.189 & 0.017 \\
    1.001 & -5.630 & 0.008 & 0.331 & 0.030 \\
    0 &  0 &  0 &  0 &  0 \\
    0 &  0 &  0 &  0 &  0 \\
    0 &  0 &  0 &  0 &  0


    



\begin{frame}

\resizebox{\linewidth}{!}{%
$\displaystyle
    \boldsymbol{Phi} =   \begin{bmatrix}
    0.827 & 0.164 &  0.009 & 0.001 &  0.005 & 0.001 &  0.005 & 0.001 &  0.004 & 0.001 &  0.004 & 0 &  0.288 & 0.006 & 0 \cr
    0.021 &  0.884 &  0    & 0.009 & 0.001 &  0.001 & 0.001 &  0 & 0 &  0 & 0 &  0 &  0.105 & 0.001 & 0 \cr
   -32.985 & -30.453 &  0.719 & 0.146 &  0.946 & 0.130 &  0.761 & 0.105 &  0.623 & 0.086 &  0.518 & 0.072 & 87.966 & 0.582 & 0.002 \cr
    3.721 & -22.557 &  0.039 &  0.775 & 0.105 &  0.098 & 0.083 &  0.078 & 0.067 &  0.064 & 0.055 &  0.053 & 30.251 & 0.148 & 0.001 \cr
    0.762 & -3.481 & 0.001 &  0.260 &  0.604 &  0.018 & 0.020 &  0.015 & 0.018 &  0.013 & 0.015 &  0.012 & -5.541 &  0.031 &  0 \cr
    
    
    
    
   -1.334 &  6.092 &  0.001 & 0.455 &  0.041 &  0.597 &  0.035 & 0.027 &  0.031 & 0.023 &  0.027 & 0.021 &  9.698 & 0.055 & 0 \cr
   -1.012 &  5.406 & 0.006 & 0.365 &  0.031 & 0.026 &  0.420 & 0.022 &  0.023 & 0.019 &  0.020 & 0.017 &  7.726 & 0.023 & 0 \cr
    1.772 & -9.461 &  0.010 &  0.639 & 0.054 &  0.046 & 0.046 &  0.433 & 0.040 &  0.034 & 0.035 &  0.030 & -13.521 &  0.041 &  0 \cr
    1.394 & -7.114 &  0.006 &  0.452 & 0.042 &  0.035 & 0.036 &  0.029 &  0.216 &  0.025 & 0.027 &  0.022 & -10.485 &  0.015 &  0 \cr
   -2.440 & 12.450 & 0.010 & 0.791 &  0.074 & 0.060 &  0.063 & 0.052 &  0.054 &  0.202 &  0.047 & 0.039 & 18.349 & 0.026 & 0 \cr
   
   
   
   
   0.572 &  3.217 & 0.005 & 0.189 &  0.017 & 0.015 &  0.015 & 0.013 &  0.013 & 0.011 &  0.166 & 0.010 &  4.417 &  0.002 & 0 \cr
    1.001 & -5.630 &  0.008 &  0.331 & 0.030 &  0.026 & 0.026 &  0.022 & 0.022 &  0.019 & 0.019 &  0.172 & -7.730 & 0.004 &  0 \cr
    0 &  0 &  0 &  0 &  0 &  0 &  0 &  0 &  0 &  0 &  0 &  0 &  0.517 &  0.005 &  0 \cr
    0 &  0 &  0 &  0 &  0 &  0 &  0 &  0 &  0 &  0 &  0 &  0 & -97.622 & 0.260 &  0.001 \cr
    0 &  0 &  0 &  0 &  0 &  0 &  0 &  0 &  0 &  0 &  0 &  0 & -3768.616 & -127.611 & 0.419
  \end{bmatrix}
$}
\end{frame}


\begin{frame}

\resizebox{\linewidth}{!}{%
$\displaystyle
    \boldsymbol{A} =   \begin{bmatrix}
   -0.288 \cr
   -0.105 \cr
   -88.008 \cr
   -30.247 \cr
    5.542 \cr
   -9.699 \cr
   -7.728 \cr
   13.523 \cr
   10.486 \cr
   -18.351 \cr
   -4.418 \cr
    7.731 \cr
    0.483 \cr
   97.622 \cr
   3768.616
  \end{bmatrix}
$}
\end{frame}






% Exemplo de como inserir figuras
\begin{figure}[htb]
	\caption{Perspectiva global do volume armazenado de dados até 2025}
	\begin{center}
		\includegraphics[scale=0.9]{template_eca_ufsc_abnt/capitulos/cap1_fig/data_growth_PT.png}
	\end{center}
	\fonte{\cite{art:data_growth_IDC}}
	\label{fig:aumento_data}
\end{figure}


% Inserir Tabelas






% ------------------------------------------------------
% Texto e referencias no texto

O uso de \gls{DWT} para compressão de sinais e imagens tem sido estudado há muito tempo. Para o campo de sinais, encontra larga utilização e estudos na área biomédica, acústica e de vibrações. \cite{art:resumo_compression_techniques} realizou uma extensa revisão dos trabalhos empregados neste campo, destacando wavelet como um método relevante nos campos de compressão de sinais cardíacos e imagens, principalmente imagiologia médica.

\cite{art:tanaka} propôs uma metodologia de compressão de sinais utilizando \gls{DWT}. Os sinais eram provenientes de uma usina termelétrica, empregados para \textit{fault diagnosis}. Uma ilustração esquemática do sistema de monitoramento das máquinas rotativas pode ser observada na Figura \ref{fig:tanaka_fault_diagnosis}.

\begin{figure}[htb]
	\caption{Esquema genérico de um sistema de monitoramento de vibração de máquinas rotativas}
	\begin{center}
		\includegraphics[scale=0.6]{template_eca_ufsc_abnt/capitulos/cap4_fig/tanaka_monitor.png}
	\end{center}
	\fonte{\cite{art:tanaka}}
	\label{fig:tanaka_fault_diagnosis}
\end{figure}

\begin{figure}[htb]
	\caption{Esquema da metodologia de compressão proposta. A metodologia proposta no atual trabalho apresenta semelhanças com a metodologia de Tanaka}
	\begin{center}
		\includegraphics[scale=0.55]{template_eca_ufsc_abnt/capitulos/cap4_fig/tanaka_wavelet_sch.png}
	\end{center}
	\fonte{\cite{art:tanaka}}
	\label{fig:tanaka_wavelet_sch}
\end{figure}

Em seu trabalho, Tanaka buscou reduzir a taxa de bits do sinal para uma distorção pré-definida pelo usuário. Para disso, \cite{art:tanaka} comparou os dados com diferentes famílias de wavelet-mãe, além de comparar a compressão de \gls{DWT} com a compressão de \gls{DCT}, levando em conta a distorção e \textit{bit-rate} (inversamente proporcional à \gls{CR}). Os sinais avaliados foram de diversos tipos de defeito em rolamentos de máquinas rotativas: desbalanceamento, falha na pista interna,  falha na pista externa e má lubrificação. 

As Figura \ref{fig:tanaka_wavelet_desempenho} apresenta os resultados de compressão \textit{versus} distorção para várias famílias de \gls{DWT}, além de comparar com a compressão utilizando \gls{DCT}. Nela, mostra-se duas condições de rolamento: rolamento em condições normais e rolamento com falha na pista externa. É possível notar o melhor desempenho geral das famílias wavelet frente à \gls{DCT} em ambos os casos, além da redução significativa de distorção para valores de \textit{bit-rate} acima de 8.

\begin{figure}[htb]
	\caption{Desempenho de compressão de diferentes wavelet-mãe e do método \gls{DCT}. (a) Rolamento em condições normais; (b) Rolamento com falha na pista externa}
	\begin{center}
		\includegraphics[scale=0.7]{template_eca_ufsc_abnt/capitulos/cap4_fig/tanaka_wavelet_desempenho.png}
	\end{center}
	\fonte{\cite{art:tanaka}}
	\label{fig:tanaka_wavelet_desempenho}
\end{figure}


Em seus trabalhos, \cite{art:staszew_vib_wavelet_compress_genetic} e \cite{art:staszew_vib_wavelet_compress_genetic2}, Wieslaw J. Staszewski realizou uma revisão teórica da teoria wavelet, além de demonstrar sua eficácia para diferentes tipos de sinais. Ele utilizou a métrica \glsxtrfull{MSE} para avaliação do erro de reconstrução. Os sinais avaliados para a avaliação de compressão foram:

\begin{itemize}
    \item Sinal estacionário contínuo
    \item Sinal contínuo não-estacionário
    \item Sinal transiente
\end{itemize}

\begin{figure}[htb]
	\caption{Avaliação temporal de diferentes sinais antes (--) e após compressão (- -): (a) Sinal estacionário contínuo, MSE=2.02\%; (b) Sinal contínuo não-estacionário, MSE=5.03\%; (c) Sinal transiente, MSE=0.20\%}
	\begin{center}
		\includegraphics[scale=0.75]{template_eca_ufsc_abnt/capitulos/cap4_fig/staszew_compress_simple.png}
	\end{center}
	\fonte{\cite{art:staszew_vib_wavelet_compress_genetic}}
	\label{fig:staszew_vib_wavelet1}
\end{figure}

O autor também mostrou a utilização das wavelets para classificação de sinais anômalos. Esta metodologia vai além do escopo deste trabalho, mas é muito utilizada no campo de \textit{health monitoring} e \textit{fault diagnosis}, um tópico já evidenciado como atual e importante no Capítulo \ref{cap:introducao} deste trabalho. O emprego desta técnica pode ainda ser aliado a outras técnicas de classificação de sinal ou detecção de anomalias a partir de redes neurais para aumentar a confiabilidade do modelo de detecção. A Figura \ref{fig:staszew_vib_wavelet2} mostra a diferença no sinal reconstruído após seleção e preservação dos parâmetros (níveis) mais relevantes para um defeito em dentes de uma engrenagem. O defeito se torna evidente, em comparação ao sinal de uma engrenagem em condições normais.

\begin{figure}[htb]
	\caption{Sinal temporal de engrenagem após seleção dos parâmetros wavelet mais importantes: (a) Engrenagem normal; (b) Engrenagem com falha nos dentes}
	\begin{center}
		\includegraphics[scale=0.7]{template_eca_ufsc_abnt/capitulos/cap4_fig/staszew_gearbox_fault2.png}
	\end{center}
	\fonte{\cite{art:staszew_vib_wavelet_compress_genetic}}
	\label{fig:staszew_vib_wavelet2}
\end{figure}

Além disso, o autor propôs um método de encontrar os coeficientes ótimos de compressão a partir de um algoritmo de otimização \glsxtrfull{GA}  \cite{art:staszew_vib_wavelet_compress_genetic2}. Para tal, utilizou os mesmos dados de falha em engrenagem mostrados na imagem \ref{fig:staszew_vib_wavelet2} e buscando preservar as bandas laterais ao segundo componente espectral. O espectro do sinal reconstruído a partir da metodologia utilizando \gls{GA} é mostrado na Figura \ref{fig:staszew_vib_wavelet3}, em conjunto com o sinal original e com o sinal reconstruído a partir de uma tecnica simples de thresholding. O zoom na região mostra o sucesso em preservar as bandas de interesse. O \gls{MSE} alcançado em (a) foi de 82.0\% avaliando de 800–1150 Hz, enquanto em (b), \gls{MSE}=24.7\%.

\begin{figure}[htb]
	\caption{Espectro de engrenagem defeituosa com: (a) thresholding simples e (b) thresholding por \gls{GA} \textit{versus} sinal original}
	\begin{center}
		\includegraphics[scale=0.8]{template_eca_ufsc_abnt/capitulos/cap4_fig/staszew_compress_GA.png}
	\end{center}
	\fonte{\cite{art:staszew_vib_wavelet_compress_genetic2}}
	\label{fig:staszew_vib_wavelet3}
\end{figure}

Nielsen demonstrou em seus trabalhos \cite{art:nielsen_biomedical_wavelet1}, \cite{art:nielsen_biomedical_wavelet2}, a utilização do método de compressão baseado em \gls{DWT} para compactação de sinais de \glsxtrfull{ECG}, utilizando um otimizador para obter a wavelet-mãe ótima, e buscando aumentar \gls{CR} e reduzir a distorção .

Os resultados da Figura \ref{fig:nielsen1} e Figura \ref{fig:nielsen2} demonstraram elevada dependência entre a família/forma de wavelet-mãe escolhida e a eficiência de compressão para determinados tipos de sinal, destacando que determinada família de wavelets podem performar a compressão melhor para certos tipos de sinais que para outros.

\begin{figure}[htb]
	\caption{Wavelets-mãe ótimas para o mesmo sinal e diferentes valores de \gls{CR}}
	\begin{center}
		\includegraphics[scale=0.65]{template_eca_ufsc_abnt/capitulos/cap4_fig/nielsen1.png}
	\end{center}
	\fonte{\cite{art:nielsen_biomedical_wavelet1} e \cite{art:nielsen_biomedical_wavelet2}}
	\label{fig:nielsen1}
\end{figure}

\begin{figure}[htb]
	\caption{Reconstrução de dois sinais diferentes}
	\begin{center}
		\includegraphics[scale=0.6]{template_eca_ufsc_abnt/capitulos/cap4_fig/nielsen2.png}
	\end{center}
	\fonte{\cite{art:nielsen_biomedical_wavelet1} e \cite{art:nielsen_biomedical_wavelet2}}
	\label{fig:nielsen2}
\end{figure}


A partir disto, \cite{art:wu_biomedic_wavelet_compress_genetic}, propôs um método de compactação de sinais de \gls{ECG} baseado em wavelet utilizando um algoritmo genético de otimização para definição dos \textit{thresholdings} em 11 camadas de decomposição wavelet. Os algoritmos genéticos de otimização são característicos pela busca de um mínimo global com base em uma população inicial de indíduos, a partir dos quais os melhores indivíduos são selecionados e cruzados, além de sofrerem mutações; desta forma, as melhores características são repassadas à geração seguinte até o valor ótimo. 

O autor utilizou a métrica \gls{PRD} como método de avaliação do erro de reconstrução (também conhecido como distorção). Ao final da otimização e após obtenção dos melhores valores para 11 sinais de treinamento, foi realizado um método de ajuste de curvas, e os valores de threshold nos 11 níveis de decomposição são determinados genericamente em função de apenas uma variável (\textit{QF}). O autor encontrou uma relação de proporcionalidade entre o aumento de \gls{CR} e \gls{PRD}, o que também foi notado em trabalhos de outros autores, como \cite{art:nielsen_biomedical_wavelet1} e \cite{art:nielsen_biomedical_wavelet2}. As Figuras \ref{fig:wu_biomedic_wavelet_compress_genetic1} e \ref{fig:wu_biomedic_wavelet_compress_genetic2} ilustram os resultados obtidos.

\begin{figure}[htb]
	\caption{Valor das escalas de quantização (\textit{thresholds}) dos coeficientes nas diferentes camadas de decomposição. As quatro curvas descrevem quatro valores $c_{-9}$ impostos: 2, 4, 8 e 16}
	\begin{center}
		\includegraphics[scale=0.75]{template_eca_ufsc_abnt/capitulos/cap4_fig/wu_coeff_level.png}
	\end{center}
	\fonte{\cite{art:wu_biomedic_wavelet_compress_genetic}}
	\label{fig:wu_biomedic_wavelet_compress_genetic1}
\end{figure}

\begin{figure}[htb]
	\caption{Relação entre \gls{CR} e o erro \gls{PRD} para a metodologia proposta (\gls{NRDPWT}-GAar) e outras duas metodologias de compressão de sinais de \gls{ECG}}
	\begin{center}
		\includegraphics[scale=0.75]{template_eca_ufsc_abnt/capitulos/cap4_fig/wu_results.png}
	\end{center}
	\fonte{\cite{art:wu_biomedic_wavelet_compress_genetic}}
	\label{fig:wu_biomedic_wavelet_compress_genetic2}
\end{figure}

Por fim, é importante ressaltar o trabalho de \cite{art:khalifa_audio_compress} e \cite{art:romano_audio_compress} no campo de compressão de sinais de áudio. Romano propôs uma compressão de sinais de áudio em tempo real, graças à um sistema processamento paralelo orientado à GPU, que atingiu alta velocidade no processamento dos dados. O tamanho do arquivo obtido ao final da compressão foi aproximadamente 10 vezes menor que o arquivo de sinal original (CR=90\%). A Figura \ref{fig:romano} traz uma comparação entre o sinal original e o sinal após compressão.


\begin{figure}[htb]
	\caption{Superposição do sinal original (azul) sobre o sinal comprimido (vermelho). À esquerda, sinal completo e à direita, detalhe de um segmento do sinal}
	\begin{center}
		\includegraphics[scale=0.45]{template_eca_ufsc_abnt/capitulos/cap4_fig/romano.png}
	\end{center}
	\fonte{\cite{art:romano_audio_compress}}
	\label{fig:romano}
\end{figure}