% ----------------------------------------------------------
\chapter{Conclusões e Trabalhos Futuros}\label{cap:conclusões}
% ----------------------------------------------------------

Foi apresentado uma metodologia para projeto de um controlador em malha fechada para sistemas aeroservoelásticos com objetivo de supressão do fenômeno de \textit{flutter}. Para isso, foi necessário modelar o sistema físico, que consiste de uma seção transversal bidimensional de asa com \textit{flap} no bordo de fuga, o que foi realizado no considerando os esforços de inércia, aerodinâmicos elásticos e externos. Em particular, os esforços aerodinâmicos dependentes da frequência reduzida foram aproximados utilizando-se de uma técnica conhecida como \glsxtrfull{RFA}, considerando quatro polos de atraso para modelar os efeitos não-estacionários. O modelo resultante foi representado no espaço de estados, o que facilitou o projeto do vetor de realimentação de estados. No presente trabalho, adotou-se o \gls{LQR} no projeto da realimentação, devido às interessantes margens de estabilidade proporcionadas.

Os resultados apresentados no capítulo anterior destacam a convergência dos resultados obtidos de característica aeroelástica do modelo com os disponíveis em literatura para o mesmo sistema. A velocidade de \textit{flutter} estimada apresentou erro percentual de apenas $2\%$ em relação ao mostrado por \textcite{book:Fung}, indicando adequação da metodologia para modelagem no domínio do tempo realizada. Esse erro pode ser justificado devido à técnica distinta para modelagem da aerodinâmica não-estacionária.

Os resultados evidenciaram a capacidade da estrutura em malha fechada proposta em estabilizar o sistema na condição de projeto, determinada como sendo o ponto de instabilidade dinâmica do sistema, ou seja, a velocidade de \textit{flutter} em malha aberta. Além disso, percebeu-se como consequência, o aumento da velocidade de \textit{flutter} do sistema em malha fechada, apesar desse aumento não ser garantido teoricamente pela formulação da lei de controle. Observou-se também o efeito desestabilizante da estrutura em malha fechada para velocidades significativamente menores do que a de projeto, introduzindo um modo de divergência no sistema. Esse efeito pode ser derivado da frequência de atuação da superfície de controle, levando o sistema a divergir de seu ponto de equilíbrio quando os esforços aerodinâmicos apresentam influência reduzida, podendo ser objeto de investigação de futuros trabalhos.

Ademais, propõe-se avaliar futuramente resultados da metodologia descrita para sistemas mais complexos, como o de uma asa 3-D, determinando-se as características modais e aerodinâmicas para as frequências de interesse e identificando os efeitos da estrutura de controle nesse sistema. Além disso, técnicas de controle mais modernas podem ser empregadas visando tratar restrições sobre o canal de entrada, que é comum em implementações práticas em sistemas de controle.

% ----------------------------------------------------------
